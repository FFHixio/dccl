% t. schneider

%!TEX TS-program = xelatex
%!TEX encoding = UTF-8 Unicode

\documentclass[11pt, letterpaper]{article}
\usepackage{fontspec} 
% DOCUMENT LAYOUT
\usepackage{geometry} 
\geometry{letterpaper, textwidth=5.5in, textheight=8.5in, marginparsep=7pt, marginparwidth=.6in}

% FONTS
\defaultfontfeatures{Mapping=tex-text} % converts LaTeX specials (``quotes'' --- dashes etc.) to unicode
\setromanfont[ItalicFont={Gentium Italic}]{Gentium}


% HEADINGS
\usepackage{sectsty} 
\usepackage{float}
\usepackage{graphicx} 
\graphicspath{{../images/}}
\usepackage[normalem]{ulem} 
\sectionfont{\rmfamily\mdseries\upshape\Large}
\subsectionfont{\rmfamily\bfseries\upshape\normalsize} 
\subsubsectionfont{\rmfamily\mdseries\upshape\normalsize} 

% PDF SETUP
% ---- FILL IN HERE THE DOC TITLE AND AUTHOR
\usepackage[dvipdfm, bookmarks, colorlinks, breaklinks, pdftitle={Goby Underwater Autonomy Project Whitepaper},pdfauthor={Toby Schneider}]{hyperref}  
\hypersetup{linkcolor=blue,citecolor=blue,filecolor=black,urlcolor=blue} 
\newcommand{\xmltag}[1]{{$<$\tt #1$>$}}


% DOCUMENT
\begin{document}


\begin{figure}[H]
\begin{minipage}[b]{0.55\linewidth}
\begin{Large}
Goby Underwater Autonomy Project Whitepaper II
\end{Large}
\vspace{0.5em}\\
\begin{footnotesize}
T. Schneider tes@mit.edu \\
Laboratory for Autonomous Marine Sensing Systems \\
MIT / WHOI Joint Program in Oceanography \& Ocean Engineering
\end{footnotesize}
\end{minipage}
\hfill
\begin{minipage}[b]{0.3\linewidth}
\begin{flushright}
\includegraphics[width=2in]{gobysoft_logo} 
\end{flushright}
\end{minipage}
\end{figure}

\vspace{0.5em}
\rule{\textwidth}{1pt}
\vspace{0.5em}

\section{What is Goby?}

Goby is special-purpose autonomy architecture (or middleware) intended to facilitate the development of artificially intelligent \textit{marine} robots.

It can be compared to systems such as LCM, Player, or ROS but is most directly influenced by MOOS.

Goby provides:

\begin{itemize}
\item An interprocess (using shared memory) and interplatform (using ethernet, serial or acoustic) communications infrastructure. Just like MOOS, many processes communicate through a central data bus.
\item A configuration tool that seamlessly integrates both command line and configuration (text) files.
\item A true object-oriented SQL logger that can support different backends from easy (SQLite3) to powerful (Postgres / MySQL).
\item \textit{(to be developed)} a Web browser tool (AJAX) for managing and examining a Goby platform.
\end{itemize}

\section{Why Goby?}

Goby is

\begin{itemize}
\item \textit{Focused}: Goby is specifically designed for marine robots, with special consideration given to the very low data rates often encountered in this environment due to the need to rely heavily on acoustic communications. \textit{(to be developed)} Goby will have the ability to easily suspend and resume for use on low powered embedded processors (such as those on gliders).
\item \textit{Collaboration}: Goby relies on a number of highly respected and actively maintained open source projects to increase its quality and ease the learning curve for developers. A short list of these libraries includes:
\begin{itemize}
\item Boost
\item Google Protocol Buffers
\item Wt
\end{itemize}
Goby is welcome to collaborators and is hosted as an open source project at \\ https://launchpad.net/goby.
\item \textit{Appealing}: Goby is simple to learn (a meaningful application can be a dozen lines of code), but this ease of entry does not sacrifice the requirements of advanced users. We have observed that marine roboticists come with a wide range of programming skills.
\item \textit{Object Oriented}: Goby, from configuration to message passing to logging, deals completely with user defined data objects (e.g. GPSMessage, TopsideCommand, EnvironmentalSummary) composed of simple types (e.g. double, string, int, unsigned int). The major modern programming languages are all object oriented and this technique appears to be  an effective means for humans to mentally conceive, implement, and troubleshoot large software projects.
\item \textit{Scientific}: \textit{(to be developed)} Goby will have the ability to store and convert between dimensional numbers (e.g. "5 meters"). Goby will have robust support for runtime and postprocessing scientific software tools (e.g. MATLAB).
\end{itemize}

\end{document}
