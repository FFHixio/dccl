\chapter{The GPS Driver example} \label{chap:gps_driver}
\MakeShortVerb{\!} % makes !foo! == !foo!
\epigraph{\textit{Man is the best computer we can put aboard a spacecraft, and the only one that can be mass produced with unskilled labor.}}{Werner von Braun}

Robots, like people, need to know where they are. The simplest way now is to use a GPS receiver. While this works only when the robot is on the surface of the ocean, it is one of the most accurate forms of positioning available and thus used as a starting point for undersea dead reckoning using Doppler Velocity Loggers (DVLs) or Inertial Measurement Units (IMUs). Therefore, reading a GPS receiver's output into a usable form for decision making is a useful and necessary ability for our marine robot. This example shows how we might do this using Goby by making an application !gps_driver_g!.

Typically we might also need to know the depth of our vehicle. This is often determined by measuring the ambient pressure. In this example, we will simulate the scalar depth reading of such a pressure sensor in !depth_simulator_g!.

Finally, it is often useful to have an aggregate of the vehicle's status that includes a snapshot of the vehicle's location, orientation, speed, heading, and perhaps other factors such as battery life and health. For this example, we call such a message a !NodeReport! and provide an application !node_reporter_g! that compiles the reports from the GPS and the depth sensor into a single message. To extend this example, we could add data from other sources, such as an inertial measurement unit (IMU) or Doppler Velocity Logger (DVL).

As the first example, the files for this example are located at the end of the chapter in section \ref{sec:gps_driver_example_code}.

\section{Reading \textit{configuration} from files and command line: DepthSimulator}

!DepthSimulator! reads a starting depth value from a configuration file and reports that value as the current depth, perturbed slightly by a random value. It's a primitive constant depth simulator, but allows us to illustrate another feature of Goby, the configuration file reader.

Goby reads configuration text files and the command line using \gls{protobuf}, in a similar manner messages are defined for passing between applications. The Goby application author provides a !.proto! file containing a protobuf message that defines all possible valid configuration values for the given application in the form of a protobuf message. Then the application instantiates a copy of this configuration message and passes it to the !goby::common::ApplicationBase! constructor with reads the configuration text file and/or command line options. If the configuration text file and/or command line options properly populate the provided proper configuration protobuf message, the message is returned to the derived class (the Goby application). Otherwise, execution of the application ends with a useful error message for the user explaining the errors involved with the passed configuration. 

Thus, for the !DepthSimulator! we define a protobuf message called !DepthSimulatorConfig!:
\begin{boxedverbatim}
message DepthSimulatorConfig
{
  required AppBaseConfig base = 1;
  required double depth = 2;
}
\end{boxedverbatim}
\resetbvlinenumber

An embedded message of type !AppBaseConfig! is always provided for configuring parameters common for all Goby applications, such as the frequency that the virtual method !loop()! is called, the name (alias) that the application is to use to communicate (if different from the compiled name), and the connection details (IP addresses, ports, etc.). The !AppBaseConfig! message is defined in goby/src/core/proto/app\_base\_config.proto.

Specifically, for our !DepthSimulator!, we only have one other configuration parameter, a double called `depth'. It is required, so our application will fail to run without a depth provided.

To use the Goby configuration reader, we create an instantation of our !DepthSimulatorConfig!
\begin{boxedverbatim}
class DepthSimulator : public goby::common::ApplicationBase
{ 
...
    static DepthSimulatorConfig cfg_;
};
\end{boxedverbatim}
\resetbvlinenumber
which must either be a global object or a static member of our class\footnote{The configuration object must be a static member so that it is instantiated \textit{before} the \texttt{goby::common::ApplicationBase} since normal members of our \texttt{DepthSimulator} class would be instantiated \textit{after} \texttt{ApplicationBase}, which would lead to trouble when \texttt{ApplicationBase} tried to use the object.}.

Then, all we must do is pass a pointer to that object to the constructor of the base class:
\begin{boxedverbatim}
    DepthSimulator()
        : goby::common::ApplicationBase(&cfg_)
\end{boxedverbatim}
\resetbvlinenumber
!goby::common::ApplicationBase! will take of the rest. To see what configuration values (with the correct syntax) can be used in our compiled !depth_simulator_g!, we can run it with the !-e! flag:

\begin{boxedverbatim}
> depth_simulator_g -e
\end{boxedverbatim}
\resetbvlinenumber

which gives us a good list of options to choose from. For many of these, the defaults will be fine for now.
\begin{boxedverbatim}
base {  #  (req)
  ethernet_address: "127.0.0.1"  # primary IP address of the 
                                 # interface to multicast over 
                                 # (opt) (default="127.0.0.1")
  multicast_address: "239.255.7.15"  # multicast IP address to 
                                     # use; 
                                     # 239.252.0.0-239.255.255.255 
                                     # is recommended (site-local). 
                                     # See also 
                                     # http://www.iana.org/assignmen
                                     # ts/multicast-addresses/multic
                                     # ast-addresses.xml (opt) 
                                     # (default="239.255.7.15")
  ethernet_port: 11142  #  (opt) (default=11142)
  platform_name: "unnamed_goby_platform"  # same as self.name for 
                                          # gobyd cfg (opt) 
                                          # (default="unnamed_goby_p
                                          # latform")
  app_name: "myapp_g"  # default is compiled name - change this 
                       # to run multiple instances (opt)
  verbosity: QUIET  # Terminal verbosity (QUIET, WARN, VERBOSE, 
                    # GUI, DEBUG1, DEBUG2, DEBUG3) (opt) 
                    # (default=QUIET)
  using_database: true  # True if using goby_database, false if 
                        # no database is to be run (opt) 
                        # (default=true)
  database_address: "127.0.0.1"  # TCP address to send database 
                                 # requests on. If omitted and 
                                 # using_database==true, 
                                 # `ethernet_address` is used (opt)
  database_port: 11142  # TCP port to send database requests on. 
                        # If omitted and using_database==true, 
                        # `ethernet_port` is used (opt)
  loop_freq: 10  # the frequency (Hz) used to run loop() (opt) 
                 # (default=10)
}
depth:   #  (req)
\end{boxedverbatim}
\resetbvlinenumber



Similarly, to see the allowed command line parameters we can run it with the !-h! (or equivalently, !--help!) flag:
\begin{boxedverbatim}
> depth_simulator_g --help
\end{boxedverbatim}
\resetbvlinenumber

which should provides output\footnote{Some of the options are removed for brevity}:
\begin{boxedverbatim}
Allowed options:

Typically given in depth_simulator_g configuration file,
but may be specified on the command line:
  --depth arg            (req)

Given on command line only:
  -c [ --cfg_path ] arg    path to depth_simulator_g configuration file 
                           (typically depth_simulator_g.cfg)
  -h [ --help ]            writes this help message
  -a [ --app_name ] arg    name to use while communicating in goby (default: 
                           ./depth_simulator_g)
  -e [ --example_config ]  writes an example .pb.cfg file
  -v [ --verbose ] arg     output useful information to std::cout. -v is 
                           verbosity: verbose, -vv is verbosity: debug1, -vvv 
                           is verbosity: debug2, -vvvv is verbosity: debug3
  -n [ --ncurses ]         output useful information to an NCurses GUI instead 
                           of stdout. If set, this parameter overrides 
                           --verbose settings.
  -d [ --no_db ]           disables the check for goby_database before 
                           publishing. You must set this if not running the 
                           goby_database.
\end{boxedverbatim}
\resetbvlinenumber

Thus, to configure !depth_simulator_g! I could create a text file (let's say depth\_simulator.cfg) with values like
\begin{boxedverbatim}
# depth_simulator.cfg
base
{
    platform_name: "AUV-1"
    loop_freq: 1
}

depth: 10.4
\end{boxedverbatim}
\resetbvlinenumber

Then, when we run !depth_simulator_g! we pass the path to the configuration file as the first command line option:
\begin{boxedverbatim}
> depth_simulator_g depth_simulator.cfg 
\end{boxedverbatim}
\resetbvlinenumber

If we didn't want to use a configuration file, we could pass the same contents of the depth\_simulator.cfg file given above on the command line instead:
\begin{boxedverbatim}
> depth_simulator_g --base 'platform_name: "AUV-1" loop_freq: 1' --depth 10.4
\end{boxedverbatim}
\resetbvlinenumber

If the same configuration values are provided in both the configuration file and on the command line, they are merged for ``repeat'' fields. For ``required'' or ``optional'' fields, the command line value overwrites the configuration file value. 

Thus, if we run
\begin{boxedverbatim}
> depth_simulator_g depth_simulator.cfg --depth 20.5
\end{boxedverbatim}
\resetbvlinenumber
!cfg_.depth()! is 20.5 since the command line provided value takes precedence.

Some commonly used configuration values have shortcuts for the command line. For example, the following two commands are equivalent ways to set the platform name:
\begin{boxedverbatim}
> depth_simulator_g --base 'platform_name: "AUV-1"'
> depth_simulator_g -p "AUV-1"
\end{boxedverbatim}
\resetbvlinenumber

Other than reading a configuration file, all !DepthSimulator! does is repeatedly write a message of type !DepthReading! (see section \ref{sec:gps_driver:depth_reading.proto}) based off a random offset to the configuration value ``depth'':
\begin{boxedverbatim}
void loop()
    {
       DepthReading reading;
       // just post the depth given in the configuration file plus a small random offset
       reading.set_depth(cfg_.depth() + (rand() % 10) / 10.0);

       glogger() << reading << std::flush;
       publish(reading);    
    }
\end{boxedverbatim}
\resetbvlinenumber

You will note that depth\_reading.proto contains an import command and a field of type `Header':
\begin{boxedverbatim}
import "goby/pb/proto/header.proto";

message DepthReading
{
  // time is in header
  required Header header = 1;
  required double depth = 2;
}
\end{boxedverbatim}
\resetbvlinenumber

`Header' (defined in goby/src/core/proto/header.proto) provides commonly used fields such as time and source / destination addressing. It is highly recommended to include this in messages sent through Goby, but not required. !goby::common::ApplicationBase! will populate any required fields in `Header' not given by DepthSimulator. For example, if the `time' is not set, !goby::common::ApplicationBase! will set the time based on the time !publish()! was called. However `time' should be set if the calling application has a better time stamp for the message than the publish time (for example, the time a sensor's sample was taken).

\section{Our first \textit{useful} application: \texttt{GPSDriver}}

!GPSDriver! doesn't introduce any new features of Goby, but it attempts to be the first non-trivial application we have seen thus far. !GPSDriver! connects to a NMEA-0183 compatible GPS receiver over a serial port, reads all the messages and parses the GGA sentence into a useful protobuf message for posting to the database.

\subsection{Configuration}
The configuration needed for !GPSDriver! all pertains to how the serial GPS receiver is connected and how it communicates:
\begin{boxedverbatim}
message GPSDriverConfig
{
  required AppBaseConfig base = 1;

  required string serial_port = 2;
  optional uint32 serial_baud = 3 [default = 4800];
  optional string end_line = 4 [default = "\r\n"];
}
\end{boxedverbatim}
\resetbvlinenumber

Note the use of defaults when they are meaningful (the NMEA-0183 specification requires carriage return (!\r!) and new line (!\n!) to signify the end of a line so this default will likely often be precisely what our users want, saving them the effort of specifying it every time).

\subsection{Protobuf Messages}
GPSDriver uses two \gls{protobuf} messages both defined in gps\_nmea.proto (see section \ref{sec:gps_driver:gps_nmea.proto}). The first (!NMEASentence!) is a parsed version of a generic NMEA-0183 message. The second (!GPSSentenceGGA!) contains a !NMEASentence! but also the parsed fields of the GGA position message. Providing the !GPSSentenceGGA! gives all subscribers of this message rapid access to useful data without parsing the original NMEA string again.

\subsection{Body}
!GPSDriver! should be straightforward to understand given what we have learned to this point. It makes use of some utilities in the goby::util libraries, especially the !goby::util::SerialClient! used for reading the serial port. These utilities are documented along with all the other Goby classes at \url{http://gobysoft.com/doc}.

Goby makes heavy use of the Boost libraries (\url{http://www.boost.org}). While you are not required to use any of Boost when developing Goby applications, it would be worth your while becoming acquainted with them. For example, the Boost Date-Time library gives a handy object oriented way to handle dates and times that far exceeds the abilities of ctime (i.e. time.h).

\section{Subscribing for \textit{multiple types}: NodeReporter}

!NodeReporter! subscribes to both the output of !DepthSimulator! (!DepthReading!) and !GPSDriver! (!GPSSentenceGGA!). Whenever either is published, a new !NodeReport! message is created as the aggregate of pieces of both messages. The !NodeReport! (defined in node\_report.proto in section \ref{sec:gps_driver:node_report.proto}) is a useful summation of the status of a given node (synonomously, platform). Because !DepthReading! and !GPSSentenceGGA! are published asynchronously, we also keep track of the delays between different parts of the NodeReport message (the !*_lag! fields). 

The !NodeReport! provides
\begin{enumerate}
\item Name of the platform
\item Type of the platform (e.g. AUV, buoy)
\item The global position of the vehicle in geodetic coordinates (latitude, longitude, depth)
\item The local position of the vehicle in a local cartesian coordinate system (x, y, z) based off the datum defined in the configuration. This is generally more useful for vehicle operators than the global fix.
\item The Euler angles of the current vehicle pose: roll, pitch, yaw (heading). 
\item The speed of the vehicle.
\end{enumerate}

In this example, we only set the first three fields given above. The others would require further sensing capability than we have in this example.

\section{Putting it all together}

First, we either need a real GPS unit or simulate one somehow. If you have a real NMEA-0183 GPS handy, by all means use it. Otherwise, I've made a fake GPS using !socat! and a log file of a real GPS (nmea.txt). This fake GPS can be run using
\begin{verbatim}
./fake_gps.sh nmea.txt
\end{verbatim}
which writes a line from nmea.txt every second to the fake serial port !/tmp/ttyFAKE!. This should be good enough for us here. If you don't have !socat!, you should be able to find it in the package manager for your Linux distribution (!sudo apt-get install socat! in Debian or Ubuntu).

Next we need to launch everything. The list is beginning to grow
\boxedverbatiminput{"@RELATIVE_CMAKE_SOURCE_DIR@/share/examples/core/ex2_gps_driver/launch_list.txt"}
\resetbvlinenumber
but fortunately we've provided a script that launches everything for you in separate terminal windows. So all you need to do is type
\begin{boxedverbatim}
./launch.sh
\end{boxedverbatim}
\resetbvlinenumber
and enjoy the magic unfold. Should you wish to modify how things are launched, just edit launch\_list.txt in goby/share/examples/core/ex2\_gps\_driver.

\section{Reading the log files (SQLite3)}

You may have noticed that everytime you run !gobyd! it creates a log file called !AUV-1_YYYYMMDDTHHMMSS_goby.db!. This is an SQLite3 \cite{sqlite} \gls{sql} database. Every variable published in Goby is written to this database. To read it, you need a tool capable of reading SQLite3 databases. One candidate is the !sqlite3! command line tool. The following will dump to your screen all the DepthReading values recorded. Using the interactive mode:
\begin{boxedverbatim}
sqlite3 AUV-1_20110304T212549_goby.db
sqlite> .mode column
sqlite> .headers ON
sqlite> SELECT * FROM DepthReading;
\end{boxedverbatim}
\resetbvlinenumber
or similarly on the command line only
\begin{boxedverbatim}
sqlite3 -header -column AUV-1_20110304T212549_goby.db "SELECT * FROM DepthReading"
\end{boxedverbatim}
\resetbvlinenumber

If a Graphical User Interface (GUI) is more your style, \url{http://www.sqlite.org/cvstrac/wiki?p=ManagementTools} has a whole list. My preference is Sqliteman, accessible in Ubuntu with !sudo apt-get install sqliteman!. Then it's just a matter of loading up the database and away you go:
\begin{boxedverbatim}
sqliteman AUV-1_20110304T212549_goby.db
\end{boxedverbatim}
\resetbvlinenumber

\section{Code} \label{sec:gps_driver_example_code}

This entire example can be browsed online at \url{http://bazaar.launchpad.net/~goby-dev/goby/1.0/files/head:/share/examples/core/ex2_gps_driver}.

\subsection{goby/share/examples/core/ex2\_gps\_driver/config.proto} \label{sec:gps_driver:config.proto}
\boxedverbatiminput{"@RELATIVE_CMAKE_SOURCE_DIR@/share/examples/core/ex2_gps_driver/config.proto"}
\resetbvlinenumber

\subsection{goby/share/examples/core/ex2\_gps\_driver/depth\_reading.proto} \label{sec:gps_driver:depth_reading.proto}
\boxedverbatiminput{"@RELATIVE_CMAKE_SOURCE_DIR@/share/examples/core/ex2_gps_driver/depth_reading.proto"}
\resetbvlinenumber

\subsection{goby/share/examples/core/ex2\_gps\_driver/depth\_simulator.cpp} \label{sec:gps_driver:depth_simulator.cpp}
\boxedverbatiminput{"@RELATIVE_CMAKE_SOURCE_DIR@/share/examples/core/ex2_gps_driver/depth_simulator.cpp"}
\resetbvlinenumber

\subsection{goby/share/examples/core/ex2\_gps\_driver/node\_report.proto} \label{sec:gps_driver:node_report.proto}
\boxedverbatiminput{"@RELATIVE_CMAKE_SOURCE_DIR@/share/examples/core/ex2_gps_driver/node_report.proto"}
\resetbvlinenumber


\subsection{goby/share/examples/core/ex2\_gps\_driver/node\_reporter.h} \label{sec:gps_driver:node_reporter.h}
\boxedverbatiminput{"@RELATIVE_CMAKE_SOURCE_DIR@/share/examples/core/ex2_gps_driver/node_reporter.h"}
\resetbvlinenumber


\subsection{goby/share/examples/core/ex2\_gps\_driver/node\_reporter.cpp} \label{sec:gps_driver:node_reporter.cpp}
\boxedverbatiminput{"@RELATIVE_CMAKE_SOURCE_DIR@/share/examples/core/ex2_gps_driver/node_reporter.cpp"}
\resetbvlinenumber


\subsection{goby/share/examples/core/ex2\_gps\_driver/gps\_nmea.proto} \label{sec:gps_driver:gps_nmea.proto}
\boxedverbatiminput{"@RELATIVE_CMAKE_SOURCE_DIR@/share/examples/core/ex2_gps_driver/gps_nmea.proto"}
\resetbvlinenumber

\subsection{goby/share/examples/core/ex2\_gps\_driver/gps\_driver.h} \label{sec:gps_driver:gps_driver.h}
\boxedverbatiminput{"@RELATIVE_CMAKE_SOURCE_DIR@/share/examples/core/ex2_gps_driver/gps_driver.h"}
\resetbvlinenumber
\subsection{goby/share/examples/core/ex2\_gps\_driver/gps\_driver.cpp} \label{sec:gps_driver:gps_driver.cpp}
\boxedverbatiminput{"@RELATIVE_CMAKE_SOURCE_DIR@/share/examples/core/ex2_gps_driver/gps_driver.cpp"}
\resetbvlinenumber

\subsection{goby/share/examples/core/ex2\_gps\_driver/gobyd.cfg} \label{sec:gps_driver:gobyd.cfg}
\boxedverbatiminput{"@RELATIVE_CMAKE_SOURCE_DIR@/share/examples/core/ex2_gps_driver/gobyd.cfg"}
\resetbvlinenumber

\subsection{goby/share/examples/core/ex2\_gps\_driver/depth\_simulator\_g.cfg} \label{sec:gps_driver:depth_simulator_g.cfg}
\boxedverbatiminput{"@RELATIVE_CMAKE_SOURCE_DIR@/share/examples/core/ex2_gps_driver/depth_simulator_g.cfg"}
\resetbvlinenumber

\subsection{goby/share/examples/core/ex2\_gps\_driver/gps\_driver\_g.cfg} \label{sec:gps_driver:gps_driver_g.cfg}
\boxedverbatiminput{"@RELATIVE_CMAKE_SOURCE_DIR@/share/examples/core/ex2_gps_driver/gps_driver_g.cfg"}
\resetbvlinenumber

\subsection{goby/share/examples/core/ex2\_gps\_driver/node\_reporter\_g.cfg} \label{sec:gps_driver:node_reporter_g.cfg}
\boxedverbatiminput{"@RELATIVE_CMAKE_SOURCE_DIR@/share/examples/core/ex2_gps_driver/node_reporter_g.cfg"}
\resetbvlinenumber

\subsection{goby/share/examples/core/ex2\_gps\_driver/nmea.txt} \label{sec:gps_driver:nmea.txt}
\boxedverbatiminput{"@RELATIVE_CMAKE_SOURCE_DIR@/share/examples/core/ex2_gps_driver/nmea.txt"}
\resetbvlinenumber

\DeleteShortVerb{\!}
