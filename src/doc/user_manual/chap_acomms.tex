\chapter{Goby-Acomms}\label{chap:acomms}
\MakeShortVerb{\!} % makes !foo! == !foo!


\section{Problem}
Acoustic communications are highly limited in throughput. Thus, it is unreasonable to expect ``total throughput'' of all communications data. Furthermore, even if total throughput is achievable over time, certain messages have a lower tolerance for delay (e.g. vehicle status) than others (e.g. CTD sample data). 

Also, in order to make the best use of this available bandwidth, messages need to be compacted to a minimal size before sending (effective encoding). To do this, Goby-Acomms provides an interface to the Dynamic Compact Control Language (DCCL\footnote{the name comes from the original CCL written by Roger Stokey for the REMUS AUVs, but with the ability to dynamically reconfigure messages based on mission need. If desired, DCCL can be configured to be backwards compatible with a CCL network using CCL message number 32}) encoder/decoder. 

\section{Dynamic Compact Control Language: DCCL} \label{sec:dccl}

DCCL allows you to take object based ``messages'' (similar to C structs) defined in the Google Protocol Buffers language and extend them to be more strictly bounded. It provides a set of default encoders for these bounded Protocol Buffers messages (now called DCCL messages) to provide a more minimal encoding than the default Protocol Buffers encoding (which is reasonably decent already, but still has too much overhead for extremely slow links). 

\subsection{Configuration: DCCLConfig}

Configuration of individual DCCL messages is done within the .proto definition. All the non-message specific available configuration for !goby::acomms::DCCLCodec! is given in its TextFormat form as:

\boxedverbatiminput{@RELATIVE_CMAKE_CURRENT_SOURCE_DIR@/includes/dccl_config.pb.cfg}
\resetbvlinenumber

\begin{itemize}
\item !crypto_passphrase!: If provided, this preshared key is used to encrypt the body of all messages using AES (Rijndael) encryption. Omit this field to turn off encryption. Note that the contents of messages received by nodes with the wrong encryption key are undefined, and such failure is not currently detected.
\item !id_codec!: The codec used to encode the message identifier (DCCL ID). !VARINT! uses a one (id: 0-127) or two byte (id: 127-32768) encoding on the wire. !LEGACY_CCL! adds another byte to DCCL messages (0x20) to the least significant (first) end to allow interoperability with REMUS CCL. Use !VARINT! unless you need to interoperate with CCL networks since you will save a byte that can be better used elsewhere.
\end{itemize}

\subsection{Configuration: Designing DCCL messages using Protocol Buffers Extensions}

A guide to designing DCCL messages is given at \url{http://gobysoft.com/doc/2.0/acomms_dccl.html} along with a full list of the DCCL extensions to google::protobuf::MessageOptions and google::protobuf::FieldOptions. Therefore, we will not replicate that information here.

\section{Time dependent priority queuing: Queue} \label{sec:queue}

Goby-Queue manages a queue for each DCCL message. When it is prompted by data by the modem, it has a priority "contest" between the queues. the queue with the current highest priority (as determined by the !value_base! and !ttl! fields) is selected. The next message in that queue is then provided to the modem to send. For modem messages with multiple frames per packet, each frame is a separate contest. Thus a single packet may contain frames from different
 queues (e.g. a rate 5 PSK packet has eight 256 byte frames. frame 1 might grab a STATUS message since that has the current highest queue. then frame 2 may grab a BTR message and frames 3-8 are filled up with CTD messages (e.g. STATUS is in blackout, BTR queue is empty)). See \url{http://gobysoft.com/doc/2.0/acomms_queue.html} for more information.

\subsection{Configuration: QueueManagerConfig}

Most of the configuration for queuing messages is done within the .proto message definition of a given DCCL message. The rest of the configuration options for !goby::acomms::QueueManager! are:

\boxedverbatiminput{@RELATIVE_CMAKE_CURRENT_SOURCE_DIR@/includes/queue_config.pb.cfg}
\resetbvlinenumber

\begin{itemize}
\item !modem_id!: A unique integer value for this particular vehicle (like a MAC address). Should be as small as possible for optimal bounding of the source and destination fields of the message. 0 is reserved for broadcast (analogous to 255.255.255.255 for IPv4).
\item !manipulator_entry!: Manipulates the queuing behavior for a given message.
\begin{itemize}
\item !protobuf_name!: String represented the Protobuf message to manipulate. Messages are named the same as !google::protobuf::Descriptor::full_name()!, which is the !package! followed by the message name, separated by dots: e.g. ``goby.acomms.protobuf.ModemTransmission''.
\item !manipulator!: One or more manipulators to apply to the queuing of this message.
\begin{itemize}
\item !NO_MANIP!: A do nothing (nop) manipulator. Same as leaving omitting this field.
\item !NO_QUEUE!: Do not queue this message when generated on this node (but messages will still be received (dequeued).
\item !NO_DEQUEUE!: Do not dequeue (receive) this message on this node (but messages will be queued). When both !NO_QUEUE! and !NO_DEQUEUE! are set, there isn't much point to having the message loaded at all.
\item !LOOPBACK!: Dequeue all instances of this message immediately upon queuing. The message is still queued and sent to its addressed destination. Often used with !PROMISCUOUS!.
\item !ON_DEMAND!: A special (advanced) feature where QueueManager assumes this queue is always full and asks for data immediately from the application upon request from the modem side. Useful for ensuring time sensitive data does not get stale.
\item !LOOPBACK_AS_SENT!: Like loopback, but rather than dequeuing upon queuing, this manipulator dequeues a copy locally upon a data request from the modem. Often used with !PROMISCUOUS!.
\item !PROMISCUOUS!: Dequeue all messages of this type even if this !modem_id! does not match the destination address.
\item !NO_ENCODE!: Same as !NO_QUEUE!, provided for backwards compatibility with Goby v1.
\item !NO_DECODE!: Same as !NO_DEQUEUE!, provided for backwards compatibility with Goby v1.
\end{itemize}
\end{itemize}
\end{itemize}


\section{Time Division Multiple Access (TDMA) Medium Access Control (MAC): AMAC} \label{sec:amac}

The AMAC unit uses time division (TDMA) to attempt to ensure a collision-free acoustic channel.

AMAC supports two variants of the TDMA MAC scheme: centralized and decentralized. As the names suggest, Centralized TDMA (!type: MAC_POLLED!) involves control of the entire cycle from a single master node, whereas each node's respective slot is controlled by that node in Decentralized TDMA. Within decentralized TDMA, Goby supports a fixed (preprogrammed) cycle (!type: MAC_FIXED_DECENTRALIZED!) that can be updated by the application. The autodiscovery mode (!type: MAC_AUTO_DECENTRALIZED!) supported in version 1 is no longer provided in version 2. To disable the AMAC, use (!type: MAC_NONE!). See \url{http://gobysoft.com/doc/2.0/acomms_mac.html} for more details.

\subsection{Configuration: MACConfig}

The !goby::acomms::MACManager! is basically a !std::list<goby::acomms::protobuf::ModemTransmission>!. Thus, its configuration is primarily such an initial list of these ``slot''s. Since !ModemTransmission! is extensible to handle different modem drivers, the AMAC configuration is also automatically extended. Some fields in !ModemTransmission! do not make sense to configure !goby::acomms::MACManager! with, so these are omitted here:

\boxedverbatiminput{@RELATIVE_CMAKE_CURRENT_SOURCE_DIR@/includes/mac_config.pb.cfg}
\resetbvlinenumber


Further details on these configuration fields: 
\begin{itemize}
\item !type!: type of Medium Access Control. See \url{http://gobysoft.com/doc/2.0/acomms_mac.html#amac_schemes} for an explanation of the various MAC schemes.
\item !slot!: use this repeated field to specify a manual polling or fixed TDMA cycle for the  !type: MAC_FIXED_DECENTRALIZED! and  !type: MAC_POLLED!. 
\begin{itemize}
\item !src!: The sending !modem_id! for this slot. Setting both src and dest to 0 causes AMAC to ignore this slot (which can be used to provide a blank slot).
\item !dest!: The receiving !modem_id! for this slot. Omit or set to -1 to allow next datagram to set destination.
\item !rate!: Bit-rate code for this slot (0-5). For the WHOI Micro-Modem 0 is a single 32 byte packet (FSK), 2 is three frames of 64 bytes (PSK), 3 is two frames of 256 bytes (PSK), and 5 is eight frames of 256 bytes (PSK).
\item !type!: Type of transaction to occur in this slot. If !DRIVER_SPECIFIC!, the specific hardware driver governs the type of this slot.
\item !slot_seconds!: The duration of this slot, in seconds.
\item !unique_id!: Integer field that can optionally be used to identify certain types of slots. For example, this allows integration of an in-band (but otherwise unrelated) sonar with the modem MAC cycle.
\end{itemize} 
\end{itemize} 


Relevant extensions of !goby::acomms::protobuf::ModemTransmission! for the WHOI Micro-Modem driver (!DRIVER_WHOI_MICROMODEM!):

\boxedverbatiminput{@RELATIVE_CMAKE_CURRENT_SOURCE_DIR@/includes/mac_mmdriver.pb.cfg}
\resetbvlinenumber

Several examples:
\begin{itemize}
\item Continous uplink from node 2 to node 1 with a 15 second pause between datagrams (node 1's configuration; same for node 2 except for !modem_id = 2!):
\begin{boxedverbatim}
modem_id: 1
type: MAC_FIXED_DECENTRALIZED
slot { src: 2  dest: 1  type: DATA  slot_seconds: 15 }
\end{boxedverbatim}
\resetbvlinenumber
\item Equal sharing for three vehicles (destination governed by next data packet):
\begin{boxedverbatim}
modem_id: 1 # 2 or 3 for other vehicles
type: MAC_FIXED_DECENTRALIZED
slot { src: 1  type: DATA  slot_seconds: 15 }
slot { src: 2  type: DATA  slot_seconds: 15 }
slot { src: 3  type: DATA  slot_seconds: 15 }
\end{boxedverbatim}
\resetbvlinenumber
\item Three vehicles with both data and WHOI Micro-Modem two-way ranging (ping):
\begin{boxedverbatim}
modem_id: 1 # 2 or 3 for other vehicles
type: MAC_FIXED_DECENTRALIZED
slot { src: 1  type: DATA  slot_seconds: 15 }
slot { 
  src: 1
  dest: 2
  type: DRIVER_SPECIFIC 
  [micromodem.protobuf.type]: MICROMODEM_TWO_WAY_PING
  slot_seconds: 5
}
slot { 
  src: 1
  dest: 3
  type: DRIVER_SPECIFIC 
  [micromodem.protobuf.type]: MICROMODEM_TWO_WAY_PING
  slot_seconds: 5
}
slot { src: 2  type: DATA  slot_seconds: 15 }
slot { src: 3  type: DATA  slot_seconds: 15 }
\end{boxedverbatim}
\resetbvlinenumber
\item One vehicle interleaving data and REMUS long-base-line (LBL) navigation pings:
\begin{boxedverbatim}
modem_id: 1
type: MAC_FIXED_DECENTRALIZED
slot { src: 1  type: DATA  slot_seconds: 15 }
slot { 
  src: 1
  dest: 2
  type: DRIVER_SPECIFIC 
  [micromodem.protobuf.type]: MICROMODEM_REMUS_LBL_RANGING
  [micromodem.protobuf.remus_lbl] {
    enable_beacons: 0xf   # enable all four: b1111
    turnaround_ms: 50
    lbl_max_range: 500 # meters
  }
  slot_seconds: 5
}
\end{boxedverbatim}
\resetbvlinenumber
\end{itemize}

\section{Abstract Acoustic (or other slow link) Modem Driver: ModemDriver} \label{sec:driver}

The ModemDriver unit provides a common interface to any modem capable of sending datagrams. It currently supports the WHOI Micro-Modem acoustic modem, UDP over the Internet, and is extensible to other acoustic (or slow link) modems. More details on the ModemDriver are available here: \url{http://gobysoft.com/doc/2.0/acomms_driver.html}.

\subsection{Configuration: DriverConfig}

Base driver configuration:

\boxedverbatiminput{@RELATIVE_CMAKE_CURRENT_SOURCE_DIR@/includes/driver_config.pb.cfg}
\resetbvlinenumber

Extensions for the WHOI Micro-Modem (!DRIVER_WHOI_MICROMODEM!):
\boxedverbatiminput{@RELATIVE_CMAKE_CURRENT_SOURCE_DIR@/includes/driver_mmdriver.pb.cfg}
\resetbvlinenumber

Extensions for the example driver (!DRIVER_ABC_EXAMPLE_MODEM!):
\boxedverbatiminput{@RELATIVE_CMAKE_CURRENT_SOURCE_DIR@/includes/driver_abc_driver.pb.cfg}
\resetbvlinenumber

Extensions for the MOOS uField driver (!DRIVER_UFIELD_SIM_DRIVER!):
\boxedverbatiminput{@RELATIVE_CMAKE_CURRENT_SOURCE_DIR@/includes/driver_ufield.pb.cfg}
\resetbvlinenumber

Extensions for the ZeroMQ/Protobuf storage driver (!DRIVER_PB_STORE_SERVER!):
\boxedverbatiminput{@RELATIVE_CMAKE_CURRENT_SOURCE_DIR@/includes/driver_pb.pb.cfg}
\resetbvlinenumber

Extensions for the UDP driver (!DRIVER_UDP!):
\boxedverbatiminput{@RELATIVE_CMAKE_CURRENT_SOURCE_DIR@/includes/driver_udp.pb.cfg}
\resetbvlinenumber


\DeleteShortVerb{\!}
