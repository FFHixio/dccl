\documentclass{article}
\usepackage{threeparttable}
\usepackage{amsmath}
\usepackage[font=footnotesize]{subfig}
\usepackage[top=1in, bottom=1in, left=1in, right=1in]{geometry}


\begin{document}
\begin{table}
\centering
\begin{threeparttable}
\caption{Definition of the DCCL Interface Description Language}
\label{tab:dccl_language}
\begin{tabular}{l|l|p{0.25\textwidth}|p{0.15\textwidth}|p{0.06\textwidth}|l}
\hline \multicolumn{6}{c}{\textbf{Message Extensions}\tnote{a}} \\ \hline
\hline \textit{Name} & \textit{Type} & \multicolumn{3}{c|}{\textit{Explanation}} & \textit{Default} \\
\hline (dccl.msg).id & int32 & \multicolumn{3}{c|}{Unique identifying integer for this message} & - \\
\hline (dccl.msg).max\_bytes & uint32 & \multicolumn{3}{c|}{Enforced upper bound for the encoded message length}  &- \\
\hline (dccl.msg).codec\_version & int32 & \multicolumn{3}{c|}{Default codec set to use (corresponds to DCCL major version)} & 2 \\
\hline (dccl.msg).codec & string & \multicolumn{3}{c|}{Name of the codec to use for encoding the base message.} & dccl.default2 \\ 
\hline (dccl.msg).codec\_group & string & \multicolumn{3}{c|}{Group of codecs to be used for encoding the fields.} & dccl.default2 \\ 
\hline \hline \multicolumn{6}{c}{\textbf{Field Extensions}\tnote{b}} \\ \hline
\hline \textit{Name} & \textit{Type} & \textit{Explanation} & \textit{Applicable Fields} & \textit{Symbol} &  \textit{Default}\\
\hline (dccl.field).precision & int32 & Decimal digits to preserve; can be negative.  & double, float & $p$ & 0 \\
\hline (dccl.field).min  & double & Minimum value that this field can contain (inclusive) & (u)intN\tnote{c}, double, float  & $x_m$ & -\\
\hline (dccl.field).max  & double & Maximum value value that this field can contain (inclusive)   & (u)intN, double, float & $x_M$ &- \\
\hline (dccl.field).max\_length  & uint32  & Maximum length (in bytes) that can be encoded & string, bytes & $L_M$ & -\\
\hline (dccl.field).max\_repeat  & uint32  & Maximum number of repeated values. & all \textit{repeated} & $r_M$ & -\\
\hline (dccl.field).codec & string  & Codec to use for this field (if omitted, the defaults given in Table~\ref{tab:dccl_enc} are used) & all & - &- \\
\hline (dccl.field).omit & bool & Do not include field in encoded message (default = false)  & all & - & False \\
\hline (dccl.field).units & Units & Physical dimensions and units system information & (u)intN, double, float & - & - \\
\hline 
\end{tabular} 
\begin{tablenotes}
\item[a] Extensions of \texttt{google.protobuf.MessageOptions}
\item[b] Extensions of \texttt{google.protobuf.FieldOptions}
\item[c] (u)intN refers to any of the integer types: int32, int64, uint32, uint64, sint32, sint64, fixed32, fixed64, sfixed32, sfixed64
\end{tablenotes}
\end{threeparttable}
\end{table}




\begin{table*}
\centering
\begin{threeparttable}
\caption{Default formulas for encoding the DCCL types.}
\label{tab:dccl_enc}
\begin{tabular}{p{0.19\textwidth}|p{0.32\textwidth}|p{0.48\textwidth}}
\hline GPB Type & Size (bits) ($q$) & Encode\tnote{a}  \\ \hline
\hline \multicolumn{3}{c}{(dccl.msg).id header (varint)} \\ \hline
\hline int32 &  $\left. 
\begin{array}{l l}
  8 & \quad \text{if $x \in [0,128)$ }\\
  16 & \quad \text{if $x \in [128, 32768)$} \\
\end{array} \right.$ & $x_{enc} =  \left\{ 
\begin{array}{l l}
  x \cdot 2 & \quad \text{if $x \in [0,128)$ }\\
  x \cdot 2 + 1 & \quad \text{if $x \in [128, 32768)$} \\
\end{array} \right.$ \\ 
\hline \hline \multicolumn{3}{c}{\textit{required} fields} \\ \hline
\hline bool & 1 & 
$x_{enc} = \left\{ 
\begin{array}{l l}
  1 & \quad \text{if $x$ is true}\\
  0 & \quad \text{if $x$ is false}\\
\end{array} \right.$
\\ 
\hline enum & $\lceil \hbox{log}_2(\sum{\epsilon_i}) \rceil$ & $x_{enc} =  i $ \\ 
\hline (u)intN & $\lceil \hbox{log}_2(x_M-x_m + 1) \rceil$ & $x_{enc} =  \left\{ 
\begin{array}{l l}
  x - x_m  & \quad \text{if $x \in [x_m,x_M]$ }\\
  0 & \quad \text{otherwise} \\
\end{array} \right.$   \\ 
\hline double, float & $\lceil \hbox{log}_2((x_M-x_m)\cdot 10^{p} + 1) \rceil$ &  $x_{enc} =  \left\{ 
\begin{array}{l l}
  \hbox{nint}((x - x_m)\cdot 10^{p}) & \quad \text{if $x \in [x_m,x_M]$ }\\
  0 & \quad \text{otherwise} \\
\end{array} \right.$   \\ 
\hline string (of length $L$) & $\lceil \hbox{log}_2(L_M + 1)\rceil + \text{min}(L, L_M) \cdot 8$ & $x_{enc} = L + \sum_{n=0}^{\text{min}(L, L_M)} x[n] \cdot 2^{8n+\lceil \hbox{log}_2(L_M + 1)\rceil}$ \\ 
\hline bytes & $L_M \cdot 8$ & $x_{enc}$ = $x$ \\ 
\hline \textit{Message} & $\sum q_{subfields}$ & $x_{enc}$ for each field of \textit{Message} appended to the previous (recursive encoding).   \\ 
\hline \hline \multicolumn{3}{c}{\textit{optional} fields} \\ \hline
\hline bool & 2 & 
$x_{enc} = \left\{ 
\begin{array}{l l}
  2 & \quad \text{if $x$ is true}\\
  1 & \quad \text{if $x$ is false}\\
  0 & \quad \text{if $x$ is not set}\\
\end{array} \right.$
\\ 
\hline enum & $\lceil \hbox{log}_2(1+\sum{\epsilon_i}) \rceil$ & $x_{enc} =  \left\{ 
\begin{array}{l l}
  i+1 & \quad \text{if $x \in \{\epsilon_i\}$ }\\
  0 & \quad \text{otherwise}\\
\end{array} \right.$   \\ 
\hline (u)intN & $\lceil \hbox{log}_2(x_M-x_m + 2) \rceil$ & $x_{enc} =  \left\{ 
\begin{array}{l l}
  x - x_m + 1  & \quad \text{if $x \in [x_m,x_M]$ }\\
  0 & \quad \text{otherwise} \\
\end{array} \right.$   \\ 
\hline double, float & $\lceil \hbox{log}_2((x_M-x_m)\cdot 10^{p} + 2) \rceil$ &  $x_{enc} =  \left\{ 
\begin{array}{l l}
  \hbox{nint}((x - x_m)\cdot 10^{p})+1 & \quad \text{if $x \in [x_m,x_M]$ }\\
  0 & \quad \text{otherwise} \\
\end{array} \right.$   \\ 
\hline string & \multicolumn{2}{c}{same as \textit{required}; empty string treated as ``not set''} \\ 
\hline bytes & $\begin{array}{l l}
  1 + L_M \cdot 8  & \quad \text{if $x$ is set} \\
  1 & \quad \text{if $x$ is not set}\\
\end{array}$ & $x_{enc} =  \left\{ 
\begin{array}{l l}
  x \cdot 2 + 1 & \quad \text{if $x$ is set }\\
  0 & \quad \text{if $x$ is not set} \\
\end{array} \right.$   \\
\hline \textit{Message} & $\begin{array}{l l}
  1 + \sum q_{subfields}  & \quad \text{if $x$ is set} \\
  1 & \quad \text{if $x$ is not set}\\
\end{array}$ & $x_{enc} = \left\{ \begin{array}{l l}
  \text{\textit{required} $x_{enc}$ appended to } 1 & \quad \text{if $x$ is set} \\
  0 & \quad \text{if $x$ is not set}\\
\end{array} \right.$ \\ 
\hline \hline \multicolumn{3}{c}{\textit{repeated} fields (of size $r$)} \\ \hline
\hline all & $\lceil \hbox{log}_2(r_M + 1)\rceil + r_M \cdot q_{required}$ & From LSB to MSB: 1. Size $r$ is encoded using the \textit{required} (u)intN encoder (with $x_m = 0, x_M = r_M$). 2. \textit{required} $x_{enc}$ is calculated for each repeated element then appended to the previous encoded element. \\ 
\hline
\end{tabular}
\begin{tablenotes}
\item Symbols (in addition to those defined in Table \ref{tab:dccl_language}):
\item $\cdot$ $x$ is the original (and decoded) value.
\item $\cdot$ $x[n]$ is the ASCII value of the nth character of the string.
\item $\cdot$ $x_{enc}$ is the encoded value. 
\item $\cdot$ $\epsilon_i$ is the $i$th child of the enumeration definition (where $i = 0, 1, 2, \ldots$), \textit{not} the value assigned to the enum (which need not be sequential).
\item $\cdot$ nint($x$) means round $x$ to the nearest integer.
\item[a] If data are out of range (e.g. $x > max$ or $x < min$), for \textit{optional} fields they are encoded as zero ($x_{enc} = 0$) and decoded as not set; for \textit{required} fields, they are encoded as the $min$ value. In the case of strings whose length exceeds $L_M$, the string is truncated to $L_M$ before encoding. Thus, care should be taken not to exceed the $min$ and $max$ values to ensure the message is losslessly decodable.
\end{tablenotes}
\end{threeparttable}
\end{table*}


\begin{table}
\centering
\caption{Base dimensions in DCCL}
\label{tab:basedims}
\begin{tabular}{l|l}
\hline Physical dimension & Symbol character \\ \hline
\hline length & L \\ 
\hline time & T \\ 
\hline mass & M \\ 
\hline plane angle & A \\ 
\hline solid angle & S \\ 
\hline current & I \\ 
\hline temperature & K \\ 
\hline amount & N \\ 
\hline luminous intensity & J \\ 
\hline information & B \\ 
\hline dimensionless & - \\ 
\end{tabular}
\end{table}

\end{document}
